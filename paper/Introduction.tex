\documentclass[a4paper,10pt]{report}
\usepackage[utf8x]{inputenc}

% Title Page
\title{Introduction}
\author{Alex en Sam}


\begin{document}
\maketitle

\begin{abstract}
\end{abstract}

\section{Introduction}

In pervasive computing, contexts are used to make applications perform a certain action. Contexts can be seen as pieces of environmental information, which the applications depend on. Sometimes contexts are not consistent due to crossreads and misreads. The correctness of the contexts can be checked by evaluating the consistency constraints associated to them. There are multiple ways to check for consistency constraints. Constaint checking techniques are split up in two classes: Non-incremental checking and incremental checking. Incremental checking consists also of two classes: Entire Constraint Checking and Partial Constraint Checking. Once inconsistent contexts have been detected, an algorithm can be used to select the correct contexts. The algoritms use specific datastructures, like ontology models, in order to resolve the inconsistencies. However, certain conflict resolving algorithms are too slow for use in practice. This paper describes a way to use an ontology model in combination with the Partial Constraint Checking algorithm in order to detect context inconsistencies fast enough, such that it can be used in practice.

\end{document}          
