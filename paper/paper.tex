\documentclass[journal]{vgtc}                % final (journal style)
%\documentclass[review,journal]{vgtc}         % review (journal style)
%\documentclass[widereview]{vgtc}             % wide-spaced review
%\documentclass[preprint,journal]{vgtc}       % preprint (journal style)
%\documentclass[electronic,journal]{vgtc}     % electronic version, journal

%% Uncomment one of the lines above depending on where your paper is
%% in the conference process. ``review'' and ``widereview'' are for review
%% submission, ``preprint'' is for pre-publication, and the final version
%% doesn't use a specific qualifier. Further, ``electronic'' includes
%% hyperreferences for more convenient online viewing.

%% Please use one of the ``review'' options in combination with the
%% assigned online id (see below) ONLY if your paper uses a double blind
%% review process. Some conferences, like IEEE Vis and InfoVis, have NOT
%% in the past.

%% Please note that the use of figures is not permitted on the first page
%% of the journal version.  Figures should begin on the second page and be
%% in CMYK or Grey scale format, otherwise, colour shifting may occur
%% during the printing process.  Papers submitted with figures on the
%% first page will be refused.

%% These three lines bring in essential packages: ``mathptmx'' for Type 1
%% typefaces, ``graphicx'' for inclusion of EPS figures. and ``times''
%% for proper handling of the times font family.

\usepackage{mathptmx}
\usepackage{graphicx}
\usepackage{times}

%% We encourage the use of mathptmx for consistent usage of times font
%% throughout the proceedings. However, if you encounter conflicts
%% with other math-related packages, you may want to disable it.

%% If you are submitting a paper to a conference for review with a double
%% blind reviewing process, please replace the value ``0'' below with your
%% OnlineID. Otherwise, you may safely leave it at ``0''.
\onlineid{0}

%% declare the category of your paper, only shown in review mode
\vgtccategory{Research}

%% allow for this line if you want the electronic option to work properly
\vgtcinsertpkg

%% In preprint mode you may define your own headline.
%\preprinttext{To appear in an IEEE VGTC sponsored conference.}

%% Paper title.

\title{Context Inconsistency Management \\Using Partial Constraint Checking}

%% This is how authors are specified in the journal style

%% indicate IEEE Member or Student Member in form indicated below
\author{Samuel Esposito ~~~~~~~~~~~~ Alexander Jurjens}


%% Abstract section.
\abstract{Thanks to the pervasive computing paradigm more and more computer systems in utility buildings and industry are context-aware. They use a representation of the world they operate in to reduce the human-computer interaction necessary for this operation. Unfortunately reasoning based on contexts is not without flaws and context inconsistencies are the main reason for context-aware applications' incongruous behavior. Detecting context inconsistencies using traditional context modeling is very computer intensive, making timely conflict resolution unfeasible in most settings.

In this paper we present two complementary approaches for improving the mitigation of context inconsistencies. First we extend the traditional ontology based context modeling approach with context lifecycles to more accurately represent the world surrounding the application. Many of the conflicts in context reasoning can simply be resolved using information about the lifecycle state of a set of specific contexts~\cite{bu:2006:CCM}. Secondly we propose partial constraint checking for more efficiently identifying and timely resolving context inconsistencies at runtime. By adding an extra constraint layer to the traditional ontology model, conflicts can be resolved by checking constraints or partial constraints only locally in the ontology. This dramatically improves performance compared to iterative evaluation of an entire ontology~\cite{xu:2010:PCC}.
Apart from resolving conflicting contexts it is also possible to represent them into the ontology model. In our paper we explore the possibilities of incorporating inconsitencies into ontologies using fuzzy OWL and discuss the consequences of this approach on standard reasoning methods~\cite{ko:2009:IOFO}.
} % end of abstract

%% Keywords that describe your work. Will show as 'Index Terms' in journal
%% please capitalize first letter and insert punctuation after last keyword
\keywords{Pervasive Computing, Ontology Model, Context Lifecycle, Inconsistency Resolution.}

%% Copyright space is enabled by default as required by guidelines.
%% It is disabled by the 'review' option or via the following command:
% \nocopyrightspace

%%%%%%%%%%%%%%%%%%%%%%%%%%%%%%%%%%%%%%%%%%%%%%%%%%%%%%%%%%%%%%%%
%%%%%%%%%%%%%%%%%%%%%% START OF THE PAPER %%%%%%%%%%%%%%%%%%%%%%
%%%%%%%%%%%%%%%%%%%%%%%%%%%%%%%%%%%%%%%%%%%%%%%%%%%%%%%%%%%%%%%%%

\begin{document}

%% The ``\maketitle'' command must be the first command after the
%% ``\begin{document}'' command. It prepares and prints the title block.

%% the only exception to this rule is the \firstsection command
\firstsection{Introduction}

\maketitle

- WHAT\\
Conflict detection\\
Conflict resolution\\
Inconsistent ontologies\\
~\\
- WHY\\
Performance\\
Automate resolution\\
Better model context\\
~\\
- HOW\\
PCC\\
CIR\\
HMM - Fuzzy set theory\\




\section{Related Work}
Pervasive or ubiquitous computing is a fast-developing discipline that has been receiving increasing attention from both researchers and software developers~\cite{xu:2010:PCC}. In the past decade, many context-aware systems have been developed, ranging from smart room environments to warehouse and supply chain management systems. Lots of effort has been put into building middleware infrastructures that handle vast amounts of sensory data and extract the context information relevant for pervasive applications. Examples of such systems are CoBrA and CORTEX~\cite{bu:2006:CCM}. Various modeling approaches have been proposed for capturing context information, of which the ontology based context model appears to be most promising for most pervasive applications~\cite{bu:2006:CCM}.

Context management for consistency however has not been adequately studied in the existing literature. None of the studies on context-awareness discusses a way for detecting context inconsistencies for reliable pervasive computing~\cite{xu:2010:PCC, bu:2006:CCM}. Even though there has been related research going on in other disciplines as artificial intelligence and software engineering, it doesn't provide adequate support for context inconsistency detection in ubuquitous computing. In addition the strategies proposed in literature for resolving context conflicts are not suited for pervasive computing. Some are based on assumptions that may not apply to general pervasive environments. Other require human participation for conflict resolution, which is usually expensive and slow for pervasive computing~\cite{xu:2010:PCC}. Finally no research has been done on the potential of fuzzy ontologies for representing context inconsistencies in the environment instead of trying to resolve them~\cite{ko:2009:IOFO}.

In this article we aim at putting a milestone for context management by presenting an efficient inconsistency detection algorithm based on a constraint language extending the traditional ontology based context model~\cite{xu:2010:PCC}. In addition we put forward a conflict resolution algorithm which is based on a context reliability heuristic~\cite{bu:2006:CCM}. Finally the use of fuzzy ontologies representing context inconsistencies as a promising alternative to conflict resolution is explored.

\section{Partial Constraint Checking}



\section{Context Inconsistency Resolution}

\section{Fuzzy Ontologies}

better modeling
training required
maybe slower
extensive research necessary

HMM
Forward-backward algorithm

\section{Discussion}

\section{Conclusion}


\bibliographystyle{abbrv}
%%use following if all content of bibtex file should be shown
%\nocite{*}
\bibliography{paper}
\end{document}
