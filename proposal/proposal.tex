\documentclass[a4paper]{article}
\usepackage{latexsym}
\usepackage[a4paper]{geometry}
\usepackage{color}
\usepackage{listings}
\usepackage[pdftex]{graphicx}
\usepackage{subfig}
\usepackage{amsmath}

%Boldface text for type writer font
\usepackage{bold-extra} %\DeclareFontShape{OT1}{cmtt}{bx}{n}{<5><6><7><8><9><10><10.95><12><14.4><17.28><20.74><24.88>cmttb10}{}

%Break words properly at the end of a line (which isn't sloppy...)
\sloppy

\title{\textsf{Context Inconsistency Management \\Using Partial Constraint Checking}}
\author{Samuel Esposito \and Alexander Jurjens}



\begin{document}
\maketitle

%% Abstract section.
\abstract{
Thanks to the pervasive computing paradigm more and more computer systems in utility buildings and industry are context-aware. They use a representation of the world they operate in to reduce the human-computer interaction necessary for their operation. Unfortunately reasoning based on contexts is not without flaws and context inconsistencies are the main reason for context-aware applications' incongruous behavior. Context consistency management is not adequately studied in existing literature and approaches for detecting and resolving context conflicts are not suited for pervasive computing.

In our paper we present two complementary approaches for improving the mitigation of context inconsistencies. 
First we present partial constraint checking for timely identifying context inconsistencies at runtime. An extra constraint layer is added to the traditional ontology based context model and conflicts can be detected by locally checking partial constraints in the ontology. This dramatically improves performance compared to iterative evaluation of an entire ontology.
Secondly we discuss the extention of the traditional ontology model with context lifecycles to more accurately represent the environment of context-aware applications. This information can then be used to estimate the relative reliability of contexts in a conflict set and discard the contexts with lowest reliabilty. 
Apart from resolving context conflicts it is also possible to represent inconsistencies into the context model itself. In our paper we present a use case in which we explore the possibilities of incorporating inconsitencies into context models using fuzzy ontologies.\\
~\\
{\bf Keywords}: Pervasive Computing, Ontology Model, Context Lifecycle, Inconsistency Resolution\\
~\\
{\bf Field of Research}: Pervasive Computing
{\bf Topic}: Context Inconsistencies
{\bf Focus / Research Question}: Improving the detection and resolution of context inconsitencies
{\bf Expected Findings / Results}: The performance of conflict detection can dramatically be improved by partial constraint checking and the resolution of context conflicts is aided by adding lifecycle information to the ontology model.
} % end of abstract




\end{document}